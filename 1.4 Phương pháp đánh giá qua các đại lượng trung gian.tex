% 1.4 Phương pháp đánh giá qua các đại lượng trung gian
\begin{frame}{1.4 PHƯƠNG PHÁP ĐÁNH GIÁ\\ QUA CÁC ĐẠI LƯỢNG TRUNG GIAN \hspace{3cm}  68. Lê Thị Thơm} 
%\framesubtitle{} 

\end{frame} 
\begin{frame}{1.4 PHƯƠNG PHÁP ĐÁNH GIÁ\\ QUA CÁC ĐẠI LƯỢNG TRUNG GIAN \hspace{3cm}  68. Lê Thị Thơm} 
%\framesubtitle{} 
\begin{block}{Phương pháp đánh giá qua các đại lượng trung gian }

\end{block}
\end{frame} 

\begin{frame}{1.4 PHƯƠNG PHÁP ĐÁNH GIÁ\\ QUA CÁC ĐẠI LƯỢNG TRUNG GIAN \hspace{3cm}  68. Lê Thị Thơm} 
%\framesubtitle{} 
\begin{block}{Phương pháp đánh giá qua các đại lượng trung gian }
\textbf{ Để chứng minh $ A \leq B$, ta đánh giá $A\leq C$ rồi chứng minh $C \leq B$ }

\end{block} 

\end{frame} 

\begin{frame}{1.4 PHƯƠNG PHÁP ĐÁNH GIÁ\\ QUA CÁC ĐẠI LƯỢNG TRUNG GIAN \hspace{3cm}  68. Lê Thị Thơm} 
%\framesubtitle{} 
\begin{block}{Phương pháp đánh giá qua các đại lượng trung gian }
\textbf{ Để chứng minh $ A \leq B$, ta đánh giá $A\leq C$ rồi chứng minh $C \leq B$ }

\end{block} 
\begin{block}{Hoặc}
\textbf{  $A \leq B $  khi và chỉ khi 
                $\begin{cases}
		 	A \leq  C  \\
		 	C \leq  B 
		 	\end{cases}$    \\}
\end{block}
\end{frame}

\begin{frame}{1.4 PHƯƠNG PHÁP ĐÁNH GIÁ\\ QUA CÁC ĐẠI LƯỢNG TRUNG GIAN \hspace{3cm}  68. Lê Thị Thơm} 
%\framesubtitle{} 

\end{frame}

\begin{frame}{1.4 PHƯƠNG PHÁP ĐÁNH GIÁ\\ QUA CÁC ĐẠI LƯỢNG TRUNG GIAN \hspace{3cm}  68. Lê Thị Thơm} 
%\framesubtitle{} 
\begin{block}{Bài toán 1:}
 Cho $a, b,c $ không âm thõa mãn $a^2 +b^2 +c^2 = 1$. Chứng minh rằng :
 
$ \dfrac{a}{b^2 + 1} + \dfrac{b}{c^2 + 1} + \dfrac{c}{a^2 + 1} \geqslant \dfrac{3}{4}  ( a \sqrt{a} + b \sqrt{b} + c \sqrt{c}) ^2 .$
 
\end{block} 
\end{frame}

\begin{frame}{1.4 PHƯƠNG PHÁP ĐÁNH GIÁ\\ QUA CÁC ĐẠI LƯỢNG TRUNG GIAN \hspace{3cm}  68. Lê Thị Thơm}
    %\framesubtitle{} 

\end{frame}

\begin{frame}{1.4 PHƯƠNG PHÁP ĐÁNH GIÁ\\ QUA CÁC ĐẠI LƯỢNG TRUNG GIAN \hspace{3cm}  68. Lê Thị Thơm}
    %\framesubtitle{} 
\begin{block}{Hướng dẫn giải bài toán 1:}
 Áp dụng hệ quả bất đẳng thức Cauchy - Schwars:
 
$ \dfrac{a_1^2}{b_1} + \dfrac{a_2^2}{b_2 } + \dfrac{a_3^2}{b_3 } + ...+ \dfrac{a_n^2}{b_n }\geqslant \dfrac{ (a_1 + a_2 +a_3 + ... +a_n)^2 }{b_1 + b_2 +b_3 + ... +b_n}.$ 

(với $ a_1 + a_2 +a_3 + ... +a_n $ là các số thực bất kì, $b_1 + b_2 +b_3 + ... +b_n$ là các số thực dương ).

Ta có 

 $\dfrac{a}{b^2 + 1} + \dfrac{b}{c^2 + 1} + \dfrac{c}{a^2 + 1} = \dfrac{a^3}{a^2b^2 + a^2} + \dfrac{b^3}{b^2c^2 + b^2} + \dfrac{c^3}{c^2a^2 + c^2}$
 
$\geqslant\dfrac{ (a \sqrt{a} + b \sqrt{b} + c \sqrt{c})^2 }{ a^2 + b^2  + c^2 + a^2b^2 + b^2c^2 + c^2a^2 } = \dfrac{ (a \sqrt{a} + b \sqrt{b} + c \sqrt{c})^2 }{ 1+ a^2b^2 + b^2c^2 + c^2a^2 } (*)$ 
\end{block} 
\end{frame}

\begin{frame}{1.4 PHƯƠNG PHÁP ĐÁNH GIÁ\\ QUA CÁC ĐẠI LƯỢNG TRUNG GIAN \hspace{3cm}  68. Lê Thị Thơm}
    %\framesubtitle{} 

\end{frame}

\begin{frame}{1.4 PHƯƠNG PHÁP ĐÁNH GIÁ\\ QUA CÁC ĐẠI LƯỢNG TRUNG GIAN \hspace{3cm}  68. Lê Thị Thơm}
    %\framesubtitle{} 
\begin{block}{Hướng dẫn giải bài toán 1:}
Ta sẽ chứng minh 

$1+ a^2b^2 + b^2c^2 + c^2a^2 \leq \dfrac{4}{3}$ 

$\Leftrightarrow a^2b^2 + b^2c^2 + c^2a^2  \leq \dfrac{1}{3}$

Theo bất đẳng thức AM - GM, ta có  

$a^4 + b^4 +c^4 \geqslant a^2b^2 + b^2c^2 + c^2a^2 $

$\Leftrightarrow a^4 + b^4 +c^4 + 2( a^2b^2 + b^2c^2 + c^2a^2) \geqslant 3(a^2b^2 + b^2c^2 + c^2a^2) $

$\Leftrightarrow (a^2 + b^2  + c^2 )^2 \geqslant 3(a^2b^2 + b^2c^2 + c^2a^2) $

$\Leftrightarrow 1 \geqslant 3(a^2b^2 + b^2c^2 + c^2a^2)  $

$\Leftrightarrow a^2b^2 + b^2c^2 + c^2a^2  \leq \dfrac{1}{3}      (**)$ 

Từ $(*)$ và $(**)$, ta có điều phải chứng minh.
\end{block}
\end{frame}

\begin{frame}{1.4 PHƯƠNG PHÁP ĐÁNH GIÁ\\ QUA CÁC ĐẠI LƯỢNG TRUNG GIAN \hspace{3cm}  68. Lê Thị Thơm} 
%\framesubtitle{} 
 
\end{frame}

\begin{frame}{1.4 PHƯƠNG PHÁP ĐÁNH GIÁ\\ QUA CÁC ĐẠI LƯỢNG TRUNG GIAN \hspace{3cm}  68. Lê Thị Thơm} 
%\framesubtitle{} 
\begin{block}{Bài toán 2:}
 Cho $a, b,c $ là các số dương . Chứng minh rằng :
 
$ \dfrac{1}{a + 2b + c} + \dfrac{1}{b + 2c + a} + \dfrac{1}{c + 2a + b} \leq \dfrac{1}{4}  ( \dfrac{1}{a} + \dfrac{1}{b} + \dfrac{1}{c})  .$
 
\end{block} 
\end{frame}

\begin{frame}{1.4 PHƯƠNG PHÁP ĐÁNH GIÁ\\ QUA CÁC ĐẠI LƯỢNG TRUNG GIAN \hspace{3cm}  68. Lê Thị Thơm} 
%\framesubtitle{} 

 \end{frame}

 \begin{frame}{1.4 PHƯƠNG PHÁP ĐÁNH GIÁ\\ QUA CÁC ĐẠI LƯỢNG TRUNG GIAN \hspace{3cm}  68. Lê Thị Thơm} 
%\framesubtitle{} 
\begin{block}{Hướng dẫn giải bài toán 2:}
 Ta có bất đẳng thức sau : $\dfrac{1}{a + b} \leq \dfrac{1}{4}  ( \dfrac{1}{a} + \dfrac{1}{b})$
 với $a,b $ là các số dương.
 
 Áp dụng bất đẳng thức trên ta có,
 
 $\dfrac{1}{a + 2b + c}= \dfrac{1}{(a+b)+(b+c)}$ \leq \dfrac{1}{4} (\dfrac{1}{a+b} + \dfrac{1}{b+c})
 
 \leq \dfrac{1}{4}[\dfrac{1}{4} (\dfrac{1}{a} + \dfrac{1}{b}) + \dfrac{1}{4} (\dfrac{1}{b} + \dfrac{1}{c})] = \dfrac{1}{16} (\dfrac{1}{a} + \dfrac{2}{b}  + \dfrac{1}{c} ) 
 
 \implies \dfrac{1}{a + 2b + c} \leq \dfrac{1}{16} (\dfrac{1}{a} + \dfrac{2}{b}  + \dfrac{1}{c} ) (*)
 
 \end{block}
 \end{frame}
 \begin{frame}{1.4 PHƯƠNG PHÁP ĐÁNH GIÁ\\ QUA CÁC ĐẠI LƯỢNG TRUNG GIAN \hspace{3cm}  68. Lê Thị Thơm} 
%\framesubtitle{} 

\end{frame}

\begin{frame}{1.4 PHƯƠNG PHÁP ĐÁNH GIÁ\\ QUA CÁC ĐẠI LƯỢNG TRUNG GIAN \hspace{3cm}  68. Lê Thị Thơm} 
%\framesubtitle{} 
\begin{block}{Hướng dẫn giải bài toán 2:}
 Tương tự ta có : 
 
 \dfrac{1}{b + 2c + a} \leq \dfrac{1}{16} (\dfrac{1}{b} + \dfrac{2}{c}  + \dfrac{1}{a} ) (**)
 
 \dfrac{1}{c + 2a + b} \leq \dfrac{1}{16} (\dfrac{1}{c} + \dfrac{2}{a}  + \dfrac{1}{b} ) (***)
 
 Cộng vế với vế của  $(*)$, $(**)$ và $(***)$ ta được: 
 
 $\dfrac{1}{a+2b+c}+\dfrac{1}{b+2c+a}+\dfrac{1}{c + 2a + b}\leq\dfrac{1}{4}(\dfrac{1}{a}+\dfrac{1}{b}+\dfrac{1}{c})$ (ĐPCM).
 
\end{block} 
\end{frame}

