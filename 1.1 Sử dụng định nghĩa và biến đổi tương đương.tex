% 1.1 Sử dụng định nghĩa và biến đổi tương đương
\begin{frame}{
1.1 SỬ DỤNG ĐỊNH NGHĨA \\ VÀ BIẾN ĐỔI TƯƠNG ĐƯƠNG \hspace{3cm}  65.Phạm Phương Thảo} 
%\framesubtitle{} 
\begin{block}{Thứ tự trên tập hợp số thực}
\pause
Trên tập số thực, với hai số $a$ và $b$ có ba trường hợp sau:

\begin{itemize}
    \item Số $a$ bằng số $b$, kí hiệu $a=b$;
    \item Số $a$ lớn hơn số $b$, kí hiệu $a>b$;
    \item Số $a$ nhỏ hơn số $b$, kí hiệu $a<b$.
\end{itemize}
\end{block} 
\pause
Số $a$ lớn hơn hoặc bằng số $b$, tức là $a>b$ hoặc $a=b$, kí hiệu $a\geq b$.

Số $a$ nhỏ hơn hoặc bằng số $b$, tức là $a<b$ hoặc $a=b$, kí hiệu $a\leq b$.
\end{frame}

\begin{frame}{
1.1 SỬ DỤNG ĐỊNH NGHĨA \\ VÀ BIẾN ĐỔI TƯƠNG ĐƯƠNG \hspace{3cm}  65.Phạm Phương Thảo} 
%\framesubtitle{} 
\begin{block}{Khái niệm bất đẳng thức}
\pause
Ta gọi hệ thức dạng $a>b$ (hay $a<0, a\geq b, a\leq b$) là bất đẳng thức và gọi $a$ là vế trái, $b$ là vế phải của bất đẳng thức.
\end{block} 
\pause
Ví dụ: 
\begin{itemize}
    \item $0<1$;
    \item $x>0$;
    \item $\dfrac{a}{b}+\dfrac{b}{a}\geq 2, \forall a,b \in \mathbb{R}^+$.
\end{itemize}
\pause
Chú ý:
\begin{itemize}
    \item Hai bất đẳng thức $1<2$ và $-3<-2$ được gọi là \textit{hai bất đẳng thức cùng chiều};
    \item Hai bất đẳng thức $1<2$ và $-2>-3$ được gọi là \textit{hai bất đẳng thức ngược chiều}.
\end{itemize}
\end{frame} 

\begin{frame}{1.1 SỬ DỤNG ĐỊNH NGHĨA \\ VÀ BIẾN ĐỔI TƯƠNG ĐƯƠNG \hspace{3cm}  65.Phạm Phương Thảo}
\begin{block}{Tính chất}
Nếu $a<b$ và $b<c$ thì $a<c$ (tính chất bắc cầu của bất đẳng thức).
\end{block}
\pause
Chú ý:
Tương tự, các thứ tự lớn hơn ($>$), lớn hơn hoặc bằng ($\geq$), nhỏ hơn hoặc bằng ($\leq$) cũng có tính chất bắc cầu.
\end{frame}

\begin{frame}{1.1 SỬ DỤNG ĐỊNH NGHĨA \\ VÀ BIẾN ĐỔI TƯƠNG ĐƯƠNG \hspace{3cm}  65.Phạm Phương Thảo}
\begin{block}{Liên hệ giữa thứ tự và phép cộng}
\pause
Khi cộng cùng một số vào hai vế của một bất đẳng thức ta được bất đẳng thức mới \textit{cùng chiều} với bất đẳng thức đã cho. 
\end{block} 
\pause
Với ba số $a, b, c$, ta có:
\begin{itemize}
    \item Nếu $a<b$ thì $a+c<b+c$;
    \item Nếu $a\leq b$ thì $a+c \leq b+c$;
    \item Nếu $a>b$ thì $a+c>b+c$;
    \item Nếu $a\geq b$ thì $a+c\geq b+c$.
\end{itemize}
\end{frame}

\begin{frame}{1.1 SỬ DỤNG ĐỊNH NGHĨA \\ VÀ BIẾN ĐỔI TƯƠNG ĐƯƠNG \hspace{3cm}  65.Phạm Phương Thảo}
\begin{block}{Liên hệ giữa thứ tự và phép nhân}
\begin{itemize}
    \item Khi nhân cả hai vế của một bất đẳng thức với cùng một \textit{số dương} ta được bất đẳng thức mới \textit{cùng chiều} với bất đẳng thức đã cho; 
\end{itemize}
\end{block}
\pause
Với ba số $a, b, c$ và $c>0$, ta có:
\begin{itemize}
    \item Nếu $a<b$ thì $ac<bc$;
    \item Nếu $a\leq b$ thì $ac \leq bc$;
    \item Nếu $a>b$ thì $ac>bc$;
    \item Nếu $a\geq b$ thì $ac\geq bc$.
\end{itemize}
\end{frame}

\begin{frame}{1.1 SỬ DỤNG ĐỊNH NGHĨA \\ VÀ BIẾN ĐỔI TƯƠNG ĐƯƠNG \hspace{3cm}  65.Phạm Phương Thảo}
\begin{block}{Liên hệ giữa thứ tự và phép nhân}
\begin{itemize}
    \item Khi nhân cả hai vế của một bất đẳng thức với cùng một \textit{số âm} ta được bất đẳng thức mới \textit{ngược chiều} với bất đẳng thức đã cho; 
\end{itemize}
\end{block}
\pause
Với ba số $a, b, c$ và $c<0$, ta có:
\begin{itemize}
    \item Nếu $a<b$ thì $ac>bc$;
    \item Nếu $a\leq b$ thì $ac \geq bc$;
    \item Nếu $a>b$ thì $ac<bc$;
    \item Nếu $a\geq b$ thì $ac\leq bc$.
\end{itemize}
\end{frame}

\begin{frame}{1.1 SỬ DỤNG ĐỊNH NGHĨA \\ VÀ BIẾN ĐỔI TƯƠNG ĐƯƠNG \hspace{3cm}  65.Phạm Phương Thảo}
\begin{block}{Ví dụ 0.}
Cho ba số thực $a, b, c$. Chứng minh rằng: $$a^2+b^2+c^2 \geq ab+bc+ca.$$
\end{block}
\pause
Ta có:
$a^2+b^2+c^2 \geq ab+bc+ca.$

$\Leftrightarrow a^2+b^2+c^2-ab-bc-ca \geq 0.$

\pause
$\Leftrightarrow 2a^2+2b^2+2c^2-2ab-2bc-2ca \geq 0.$
\pause

$\Leftrightarrow (a^2-2ab+b^2)+(b^2-2bc+c^2)+(c^2-2ca+a^2) \geq 0.$

$\Leftrightarrow (a-b)^2+(b-c)^2+(c-a)^2 \geq 0.$

\pause
Bất đẳng thức luôn đúng do đó ta có được điều phải chứng minh.

Dấu "$=$" xảy ra khi và chỉ khi $a=b=c$.
\end{frame}

\begin{frame}{1.1 SỬ DỤNG ĐỊNH NGHĨA \\ VÀ BIẾN ĐỔI TƯƠNG ĐƯƠNG \hspace{3cm}  65.Phạm Phương Thảo}
\begin{block}{Ví dụ 1.}
Cho ba số thực $a, b, c$. Chứng minh rằng: $$3(a^2+b^2+c^2) \geq (a+b+c)^2 \geq 3(ab+bc+ca).$$
\end{block}
\pause
Ta có: $3(ab+bc+ca).$

\pause

$=ab+bc+ca+2(ab+bc+ca).$

\pause
$\leq a^2+b^2+c^2+2(ab+bc+ca).$

$=(a+b+c)^2.$

\pause
$\leq a^2+b^2+c^2+2(a^2+b^2+c^2).$

$=3(a^2+b^2+c^2).$

Ta được điều phải chứng minh.

Dấu "$=$" xảy ra khi và chỉ khi $a=b=c$.
\end{frame}

\begin{frame}{1.1 SỬ DỤNG ĐỊNH NGHĨA \\ VÀ BIẾN ĐỔI TƯƠNG ĐƯƠNG \hspace{3cm}  65.Phạm Phương Thảo}
\begin{block}{Bài 1.}
Cho ba số thực dương $x, y, z$ sao cho $x+y+z=1$. Chứng minh rằng:
$$\sqrt{6x+1}+\sqrt{6y+1}+\sqrt{6z+1}\leq 3\sqrt{3}.$$
    
\end{block}
\pause
Đặt $a=\sqrt{6x+1}; b=\sqrt{6y+1}; c=\sqrt{6z+1}$; $a, b, c>0$.

\pause
Khi đó, ta có:

$a^2+b^2+c^2=6x+1+6y+1+6z+1=6(x+y+z)+3=6.1+3=9$.

\pause
Ta cần chứng minh: $a+b+c\leq 3\sqrt{3}$.

\pause
Sử dụng kết quả ở \textbf{Ví dụ 1.} Ta có:

$3(a^2+b^2+c^2) \geq (a+b+c)^2.$
\pause

$\Leftrightarrow 3.9 \geq (a+b+c)^2.$ 

Mà $a, b, c>0$ nên ta có: $3\sqrt{3}\geq a+b+c$.
\end{frame}
