%3: Hàm lồi và bất đẳng thức Jensen
\begin{frame}{{Bất đẳng thức Jensen} \hspace{6.5cm} 76.Dương Cẩm Tú} 
%\framesubtitle{} 
\begin{block}{Định lý Jensen}
            Nếu $y=f(x)$ là hàm lồi trong khoảng $(a;b)$ thì với mọi $x_1;x_2;...x_n \in (a;b)$ và mọi số thực $\alpha_1,\alpha_2,..., \alpha_n\geq0$, $\displaystyle \sum_{i=1}^{n}\alpha_i=1,n\geq2$, ta có bất đẳng thức:
        \begin{center}
                $\alpha_1 f(x_1)+\alpha_2 f(x_2)+...+ \alpha_n f(x_n)\geq f(\alpha_1 x_1+\alpha_2 x_2+...+ \alpha_n x_n).$
        \end{center}
\end{block} 
\pause 
\textbf {\textit{Chứng minh:}}\\
\pause 
Ta chứng minh bất đẳng thức bằng phương pháp quy nạp theo $n$.\\
\pause 
\vspace{0,25cm}
Với $n=2$ bất đẳng thức đúng theo định nghĩa.\\
\pause 
\vspace{0,25cm}
Giả sử bất đẳng thức đúng với $n\geq2$. Ta chứng minh bất đẳng thức đúng cho $n+1$.\\
\pause 
Ta xét $x_1,x_2,...,x_n,x_{n+1}\in(a;b)$ và các số thực $\alpha_1,\alpha_2,..., \alpha_{n+1}\geq0$, $\displaystyle \sum_{i=1}^{n}\alpha_i=1$
\end{frame}
\begin{frame}{{Bất đẳng thức Jensen} \hspace{6.5cm} 76.Dương Cẩm Tú} 
%\framesubtitle{} 
Từ giả thiết quy nạp ta có bất đẳng thức:\\
\pause 
\vspace{0,25cm}
$f(\alpha_1 x_1+...+ \alpha_n x_n+\alpha_{n+1} x_{n+1}) \leq \alpha_1 f(x_1)+...+ (\alpha_n+\alpha_{n+1}) f\Big(\displaystyle\frac{\alpha_n}{\alpha_n+\alpha_{n+1}}x_n+\displaystyle\frac{\alpha_{n+1}}{\alpha_n+\alpha_{n+1}}x_{n+1}\Big).$\\
\pause 
\vspace{0,5cm}
Vì $y=f(x)$ là hàm lồi nên\\
\vspace{0,5cm}
$f\Big(\displaystyle\frac{\alpha_n}{\alpha_n+\alpha_{n+1}}x_n+\displaystyle\frac{\alpha_{n+1}}{\alpha_n+\alpha_{n+1}}x_{n+1}\Big) \leq \displaystyle\frac{\alpha_n}{\alpha_n+\alpha_{n+1}} f(x_n)+ \displaystyle\frac{\alpha_{n+1}}{\alpha_n+\alpha_{n+1}} f(x_{n+1})$.\\
\vspace{0,5cm}
\pause 
Vậy $f\Big(\displaystyle\sum_{i=1}^{n+1}\alpha_i x_i\Big)\leq \sum_{i=1}^{n+1}\alpha_if(x_i)$ và ta có điều phải chứng minh.
\end{frame}
%3: Hàm lồi và bất đẳng thức Jensen
\begin{frame}{{Bất đẳng thức Jensen} \hspace{6.5cm} 76.Dương Cẩm Tú} 
%\framesubtitle{} 

\begin{block}{Bài toán đề xuất} 
\textbf{Bài toán}: Cho $n$ số thực dương $a_1,a_2,...,a_n$. Chứng minh rằng:\\
\begin{center}
 $\displaystyle\dfrac{1}{a_1}+\dfrac{1}{a_2}+...+\dfrac{1}{a_n}\geq \dfrac{n^2}{a_1+a_2+...+a_n}.$\\   
\end{center}
\end{block} 
\textbf{\textit{Chứng minh}}
Xét hàm số $f(x)=\frac{1}{x}$ với $x\in (0;+\infty)$.\\
\pause 
Khi đó $f(x)$ là hàm lồi trên khoảng $(0;+\infty)$.\\
\pause 
Áp dụng bất đẳng thức Jensen ta có:\\
\pause 
\vspace{0,15cm}
$f\Big( \displaystyle\frac{a_1+a_2+...+a_n}{n}\Big) \leq \frac{1}{n}.[f(a_1)+f(a_2)+...+f(a_n)].$\\
\pause 
\vspace{0,15cm}
$\Leftrightarrow \displaystyle\frac{n}{a_1+a_2+...+a_n}\leq \frac{1}{n}.\Big[ \frac{1}{a_1}+\frac{1}{a_2}+...+\frac{1}{a_n}\Big].$\\
\pause 
\vspace{0,15cm}
$\Leftrightarrow \displaystyle\frac{1}{a_1}+\frac{1}{a_2}+...+\displaystyle\frac{1}{a_n}\geq \frac{n^2}{a_1+a_2+...+a_n}.$
\end{frame}

\begin{frame}{{Bất đẳng thức Jensen} \hspace{6.5cm} 76.Dương Cẩm Tú} 
%\framesubtitle{} 
\begin{block}{Nhận xét từ bài toán đề xuất}
Bài toán trên là cơ sở để thực hiện và sáng tạo các bài toán BĐT ở phổ thông:\\
  \begin{itemize}
    \item $n=2$ ta được bất đẳng thức $\frac{1}{x}+\frac{1}{y}\geq \frac{4}{x+y}$ với $x,y>0$.
    \item $n=3$ ta được bất đẳng thức $\frac{1}{x}+\frac{1}{y}+\frac{1}{z}\geq \frac{9}{x+y+z}$ với $x,y,z>0$.
\end{itemize}
\end{block} 
\end{frame}
\begin{frame}{{Bất đẳng thức Jensen} \hspace{6.5cm} 76.Dương Cẩm Tú} 
%\framesubtitle{} 
\begin{block}{Bài toán đề xuất} 
 \textbf{Bài toán}: Cho $n$ số thực dương $a_1,a_2,...,a_n$. Chứng minh rằng:\\
\begin{center}
$\displaystyle\frac{1}{a_1}+\frac{1}{a_2}+...+\frac{1}{a_n}\geq \frac{n^2}{a_1+a_2+...+a_n}. $\\
\end{center}
\end{block} 
\vspace{0,25cm}
\pause
\begin{block}{Các bài toán ví dụ:}
  \begin{itemize}
    \item \textbf{Bài 1:} Cho $a,b,c$ là những số thực dương thỏa mãn $a+b+c=1$. Chứng minh rằng:\\
    \begin{center}
         $\displaystyle\frac{a}{\sqrt{1-a}}+\frac{b}{\sqrt{1-b}}+\frac{c}{\sqrt{1-c}}\geq \frac{\sqrt{6}}{2}$.
    \end{center}
    \pause
    \item \textbf{Bài 2: (SGT trang 129)} Cho $x_1,x_2,...,x_n>0$. Chứng minh rằng:\\
    \begin{center}
          $\displaystyle \sum_{i=1}^{n} \frac{1}{1+x_i}\leq \frac{n}{1+\sqrt[n]{x_1x_2...x_n}}$.  
    \end{center}
\end{itemize}
\end{block} 
\end{frame}
\begin{frame}{{Bất đẳng thức Jensen} \hspace{6.5cm} 76.Dương Cẩm Tú} 
%\framesubtitle{} 


\textbf{Bài 1:} Cho $a,b,c$ là những số thực dương thỏa mãn $a+b+c=1$. Chứng minh rằng:\\
\begin{center}
      $\displaystyle\frac{a}{\sqrt{1-a}}+\frac{b}{\sqrt{1-b}}+\frac{c}{\sqrt{1-c}}\geq \frac{\sqrt{6}}{2}$.\\
    \pause
    \end{center}

    \textbf{Lời giải}
\begin{itemize}
 \item Xét hàm số $f(x)=\frac{x}{\sqrt{1-x}}$ với $x\in(0;1)$.\\
\pause
 \item Ta có $f(x)=\displaystyle\frac{x}{\sqrt{1-x}}$ xác định, liên tục trong khoảng $(0;1)$ và có đạo hàm $f'(x)=\displaystyle\frac{2-x}{2\sqrt{(1-x)^3}}$ cũng liên tục trong khoảng $(0;1)$.\\
\pause
\vspace{0,25cm}
 \item Ta có $f''(x)=\displaystyle\frac{4-x}{8\sqrt{(1-x)^5}}>0; \forall x\in (0;1)$
$\Rightarrow f(x)$ là hàm lồi trên $(0;1)$.\\
\pause
 \item Áp dụng bất đẳng thức Jensen với $a,b,c>0$ ta có:\\
$\frac{1}{3}f(a)+\frac{1}{3}f(b)+\frac{1}{3}f(c)\geq f\Big(\frac{a+b+c}{3}\Big)$
$\Rightarrow f(a)+f(b)+f(c)\geq 3.f\Big(\frac{1}{3}\Big)=\frac{\sqrt{6}}{2}.$\\
\pause
Hay $\displaystyle\frac{a}{\sqrt{1-a}}+\frac{b}{\sqrt{1-b}}+\frac{c}{\sqrt{1-c}}\geq \frac{\sqrt{6}}{2}$.
Vậy bất đẳng thức được chứng minh.
\end{itemize}
\end{frame}
\begin{frame}{{Bất đẳng thức Jensen} \hspace{6.5cm} 76.Dương Cẩm Tú} 
%\framesubtitle{} 
\textbf{Bài 2: (SGT trang 129)} Cho $x_1,x_2,...,x_n>0$. Chứng minh rằng:\\
\begin{center}
     $\displaystyle \sum_{i=1}^{n} \frac{1}{1+x_i}\leq \frac{n}{1+\sqrt[n]{x_1x_2...x_n}}$.\\
    \pause
    \textbf{Lời giải}\\
\end{center}
\begin{itemize}

    \item Xét hàm số $f(y)=\displaystyle\frac{1}{1+e^y}$ với $y\in (0;+\infty)$.\\
\pause
\vspace{0,25cm}
    \item Ta có: $f'(y)=-\displaystyle\frac{e^y}{(1+e^y)^2}$\\
\pause
\vspace{0,25cm}
    $\Rightarrow \displaystyle f''(y)=\frac{e^y(e^y-1)}{(1+e^y)^3}>0$ với mọi $y\in(0;+\infty)$.\\
\pause
\vspace{0,25cm}
$\Rightarrow \displaystyle f(y)$ là hàm lồi trong $(0;+\infty)$\\

\end{itemize}
\end{frame}
\begin{frame}{{Bất đẳng thức Jensen} \hspace{6.5cm} 76.Dương Cẩm Tú} 
%\framesubtitle{}
\begin{itemize}
 \item Áp dụng bất đẳng thức Jensen ta có:\\
\pause
\vspace{0,25cm}
 $\displaystyle f\Big( \frac{y_1+y_2+...+y_n}{n}\Big) \leq \frac{1}{n}. [f(y_1)+f(y_2)+...+f(y_n)]$với $y_i>0, i= \overline{1,n}$. $(1)$\\
\item Lấy $y_i=\ln{x_i},i= \overline{1,n} \Rightarrow y_i>0$ do $x_i>1$\\ 
 \item Từ $(1)$ ta có:\\
\pause
\vspace{0,25cm}
$\displaystyle\frac{1}{1+e^{\frac{\ln{x_1}+\ln{x_2}+...+\ln{x_n}}{n}}}\leq \frac{1}{n}.\Big( \frac{1}{1+e^{\ln x_1}}+\frac{1}{1+e^{\ln x_2}}+...+\frac{1}{1+e^{\ln x_n}}\Big)$\\
\pause
\vspace{0,25cm}
$\Leftrightarrow\displaystyle\frac{1}{1+\sqrt[n]{e^{\ln(x_1x_2...x_n)}}}\leq \frac{1}{n}.\Big( \frac{1}{1+x_1} +\frac{1}{1+x_2} +...+ \frac{1}{1+x_n}\Big)$\\
\pause
\vspace{0,25cm}
$\Leftrightarrow \displaystyle \sum_{i=1}^{n} \frac{1}{1+x_i}\leq \frac{n}{1+\sqrt[n]{x_1x_2...x_n}}$ (đpcm)
\end{itemize}
\end{frame}
\begin{frame}{{Bất đẳng thức Jensen} \hspace{6.5cm} 76.Dương Cẩm Tú} 
%\framesubtitle{}
\begin{block}{Nhận xét: Dựa vào bài toán trên, có thể sử dụng làm bài tập hoặc một bổ đề ở các bài toán phổ thông}
\begin{itemize}
    \item Nếu $n=2$ thì ta có BĐT sau: $\frac{1}{1+x_1}+\frac{1}{1+x_2}\geq \frac{2}{1+\sqrt{x_1x_2}}$.\\
    \item Nếu $n=3$ thì ta có BĐT sau: $\frac{1}{1+x_1}+\frac{1}{1+x_2}+\frac{1}{1+x_3}\geq \frac{3}{1+\sqrt[3]{x_1x_2x_3}}$.
\end{itemize}
\end{block}
\begin{block}{Các dạng bài áp dụng bất đẳng thức Jensen}
\begin{itemize}
    \item Đa số các bài toán khá tường minh để thuận tiện cho việc chọn hàm lồi và sử dụng bất đẳng thức Jensen.
    \item Trong trường hợp chưa thể nhận biết rõ ràng hàm lồi cần xác định, chúng ta có thể thử một số phương án với hàm $e^x; \ln x;...$
\end{itemize}

\end{block}
\end{frame}