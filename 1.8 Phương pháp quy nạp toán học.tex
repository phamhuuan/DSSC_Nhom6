% 1.8 Phương pháp quy nạp toán học
\begin{frame}{1.8 PHƯƠNG PHÁP QUY NẠP TOÁN HỌC \hspace{2cm}  72. Hà Anh Thư} 
%\framesubtitle{} 

\begin{block}{Nội dung phương pháp quy nạp toán học}
Có thể chứng minh một mệnh đề phụ thuộc vào số tự nhiên $n$ là đúng với mọi $n \ge n_0$ bằng phương pháp quy nạp toán học theo ba bước:
\pause
\begin{enumerate}
    \item Kiểm tra cụ thể mệnh để đúng với mọi $n = n_0$.
    \pause
    \item Giả sử mệnh đề đúng tới $n = k \in \mathbb{N}, k \ge n_0$.
    \pause
    \item Chứng minh mệnh đề đúng với $n = k + 1$ bằng cách sử dụng giả thiết và điều đã giả sử ở bước thứ hai (được gọi là giả thiết quy nạp). Khi đó kết luận mệnh đề đúng với mọi $n \ge n_0$.
\end{enumerate}
\end{block} 
\end{frame}
\begin{frame}{1.8 PHƯƠNG PHÁP QUY NẠP TOÁN HỌC \hspace{2cm}  72. Hà Anh Thư} 
\begin{block}{Ví dụ: Bài tập 1 trang 121}
Chứng minh rằng với $x > 0$ và số nguyên dương $n \ge 1$ ta có:
\begin{center}
    $e^x > 1 + \dfrac{x}{1!} + \dfrac{x^2}{2!} + ... + \dfrac{x^n}{n!}$
\end{center}
\pause
\begin{center}
    Giải:   
\end{center}
Xét $f_n(x) = e^x - 1 - \dfrac{x}{1!} - \dfrac{x^2}{2!} - ... - \dfrac{x^n}{n!}$.

Cần chứng minh $f_n(x) > 0 \forall x > 0, n \ge 1$.
\pause
\begin{enumerate}
    \item Với $n = 1, f_1(x) = e^x - 1- x \Rightarrow f_1'(x) = e^x - 1 > 0$ \\
    \pause
    Hàm số $f_1(x)$ đồng biến $\forall x > 0 \Rightarrow f_1(x) > f_1(0) = 0$. (đúng)
    \pause
    \item Giả sử đúng với $n = k$.
\end{enumerate}
\end{block}
\end{frame}
\begin{frame}{1.8 PHƯƠNG PHÁP QUY NẠP TOÁN HỌC \hspace{2cm}  72. Hà Anh Thư} 
\begin{block}{Ví dụ: Bài tập 1 trang 121}
\begin{enumerate}
    \setcounter{enumi}{2}
    \item Với $n = k + 1$. Cần chứng minh:
    \begin{center}
        $f_{k + 1}(x) = e^x - 1 - \dfrac{x}{1!} - \dfrac{x^2}{2!} - ... - \dfrac{x^k}{k!} - \frac{x^{k + 1}}{(k + 1)!} > 0$
    \end{center}
    \pause
    Thật vậy,
    $f_{k + 1}'(x) = e^x - 1 - \dfrac{x}{1!} - \dfrac{x^2}{2!} - ... - \dfrac{x^k}{k!} = f_k(x) > 0$ (Theo giả thiết quy nạp).
    \pause
    Hàm số $f_{k + 1}(x)$ đồng biến $\forall x > 0, n \ge 1 \Rightarrow f_{k + 1}(x) > f_{k + 1}(0) = 0$. (đúng) \\
    Vậy, ta có điều phải chứng minh.
\end{enumerate}
\end{block}

\end{frame}

\begin{frame}{1.8 PHƯƠNG PHÁP QUY NẠP TOÁN HỌC \hspace{2cm}  72. Hà Anh Thư} 
\begin{block}{Các bài toán phổ thông sử dụng phương pháp quy nạp toán học}
\textbf{Ví dụ 1:} Chứng minh bất đẳng thức $2^n > 2n + 1$ $(1)$ luôn đúng với mọi số tự nhiên $n \ge 3$ \\
\pause
\begin{center}
    Giải:
\end{center}
\begin{itemize}
    \item Khi $n = 3$ ta có $2^3 = 8 > 2.3 + 1 = 7$.
    \pause
    \item Giả sử $(1)$ đúng với $n = k \ge 3 (k \in \mathbb{N}) \Rightarrow 2^k > 2k + 1$ $(2)$. \\
        \pause
        $\Rightarrow$ Ta cần chứng minh $(2)$ đúng với $n = k + 1$. \\
        \pause
        $\Rightarrow 2^{k + 1} > 2(k + 1) + 1 \Leftrightarrow 2^{k + 1} > 2k + 3$ \\
        \pause
    \item Nhân cả 2 vế của $(2)$ với $2$ ta có: \\
        $2.2^k > 2k + 2k + 2 \Leftrightarrow 2^{k + 1} > 2k + 2k + 2$ $(3)$ \\
        \pause
        Vì $k \ge 3$ nên $2k \ge 6$. Do đó $(3) \Leftrightarrow 2^{k + 1} > 2k + 6 + 2 \Rightarrow 2^{k + 1} > 2k + 3$ \\
        $\Rightarrow$ Bất đẳng thức đúng với $n = k + 1$ $\Rightarrow$ Điều cần chứng minh.
\end{itemize}
\end{block}

\end{frame}

\begin{frame}{1.8 PHƯƠNG PHÁP QUY NẠP TOÁN HỌC \hspace{2cm}  72. Hà Anh Thư} 
\begin{block}{Các bài toán phổ thông sử dụng phương pháp quy nạp toán học}
\textbf{Ví dụ 2:} (Bất đẳng thức Bernoulli) Chứng minh bất đẳng thức $(1 + x)^n > 1 + nx$ $(1)$ luôn đúng với mọi số tự nhiên $x \ge -1, x \ne 0$ và với mọi số tự nhiên $n \ge 2$. \\
\pause
\begin{center}
    Giải:
\end{center}
\begin{itemize}
    \item Khi $n = 2$ ta có $(1 + x)^2 > 1 + 2x \Leftrightarrow x^2 > 0$ đúng do $x \ne 0$.
    \pause
    \item Giả sử $(1)$ đúng với $n = k \ge 2 (k \in \mathbb{N}) \Rightarrow (1 + x)^k > 1 + kx$ $(2)$. \\
        \pause
        $\Rightarrow$ Ta cần chứng minh $(2)$ đúng với $n = k + 1$. \\
        \pause
        Ta có $(1 + x)^{k + 1} = (1 + x)^k(1 + x) > (1 + kx)(1 + x)$ \\
        \pause
        Ta có $(1 + kx)(1 + x) =  1 + (k + 1)x + kx^2 > 1 + (k + 1)x$ \\
        $\Rightarrow$ Bất đẳng thức đúng với $n = k + 1$ $\Rightarrow$ Điều cần chứng minh.
\end{itemize}
\end{block}

\end{frame}

\begin{frame}{1.8 PHƯƠNG PHÁP QUY NẠP TOÁN HỌC \hspace{2cm}  72. Hà Anh Thư} 
\begin{block}{Các bài toán phổ thông sử dụng phương pháp quy nạp toán học}
\textbf{Ví dụ 3:} Chứng minh bất đẳng thức $(n!)^2 \ge n^n$ $(1)$ luôn đúng với mọi $n \in \mathbb{N}^*$. \\
\pause
\begin{center}
    Giải:
\end{center}
\begin{itemize}
    \item Trước hết ta chứng minh $n^n \ge (n + 1)^{n - 1}$ $(2)$ luôn đúng với mọi $n \in \mathbb{N}^*$ \\
    \pause
    \item Khi $n = 1$ ta có $1^1 \ge 2^0 \Leftrightarrow 1 \ge 1$ đúng. \\
    \pause
    \item Giả sử $(2)$ đúng với $n = k \in \mathbb{N}^* \Leftrightarrow k^k \ge (k + 1)^{k - 1} \Leftrightarrow \left(\dfrac{k}{k + 1}\right)^k \ge \dfrac{1}{k + 1}$ $(3)$ \\
        \pause
        $\Rightarrow$ Ta cần chứng minh $(3)$ đúng với $n = k + 1$. \\
        \pause
        Ta có $k^2 + 2k + 1 \ge k^2 + 2k \Rightarrow \dfrac{k + 1}{k + 2} \ge \dfrac{k}{k + 1} \Rightarrow \left(\dfrac{k + 1}{k + 2}\right)^k \ge \left(\dfrac{k}{k + 1}\right)^k \ge \dfrac{1}{k + 1}$ \\
        $\Rightarrow \left(\dfrac{k + 1}{k + 2}\right)^{k + 1} \ge \dfrac{1}{k + 1} \Rightarrow n^n \ge (n + 1)^{n - 1}$ đúng với $n \in \mathbb{N}^*$
\end{itemize}
\end{block}

\end{frame}

\begin{frame}{1.8 PHƯƠNG PHÁP QUY NẠP TOÁN HỌC \hspace{2cm}  72. Hà Anh Thư} 
\begin{block}{Các bài toán phổ thông sử dụng phương pháp quy nạp toán học}
\begin{itemize}
    \item Xét $(1)$ khi $n = 1$ ta có $(1!)^2 \ge 1^1 \Leftrightarrow 1 \ge 1$ đúng.
    \pause
    \item Giả sử $(1)$ đúng với $n = k \in \mathbb{N}^* \Leftrightarrow (k!)^2 \ge k^k$ $(4)$ \\
        \pause
        $\Rightarrow$ Ta cần chứng minh $(4)$ đúng với $n = k + 1$ $\Leftrightarrow \left[(k + 1)!\right]^2 \ge (k + 1)^{k + 1}$ \\
        \pause
        Ta có $\left[(k + 1)!\right]^2 = \left[k!(k + 1)\right]^2 = (k!)^2(k + 1)^2 \ge k^k(k + 1)^2$ \\
        \pause
        Áp dụng $(2)$ ta có $k^k(k + 1)^2 > (k + 1)^{k - 1}(k + 1)^2 = (k + 1)^{k + 1}$ \\
        $\Leftrightarrow$ Bất đẳng thức đúng với $n = k + 1$ $\Rightarrow$ Điều cần chứng minh.
\end{itemize}
\end{block}

\end{frame}

\begin{frame}{1.8 PHƯƠNG PHÁP QUY NẠP TOÁN HỌC \hspace{2cm}  72. Hà Anh Thư} 
\begin{block}{Các bài toán phổ thông sử dụng phương pháp quy nạp toán học}
\textbf{Ví dụ 4:} (Vô địch Toán Matxcova 1984) Cho $x_1, x_2, ..., x_n$ là $n$ số không âm $(n \in \mathbf{Z}, n \ge 4)$, tổng của chúng bằng $1$. Chứng minh rằng $x_1x_2 + x_2x_3 + ... + x_nx_1 \le \dfrac{1}{4}$ $(1)$ \\
\pause
\begin{center}
    Giải:
\end{center}
\begin{itemize}
    \item $(1) \Leftrightarrow (x_1 + x_2 + ... + x_n)^2 \ge 4(x_1x_2 + x_2x_3 + ... + x_nx_1)$
    \pause
    \item Khi $n = 4$ ta có $(x_1 + x_2 + x_3 + x_4)^2 \ge 4$. $\Leftrightarrow (x_1 - x_2 + x_3 - x_4)^2 \ge 0$ đúng với $n = 4$.
    \pause
    \item Giả sử $(1)$ đúng với $n = k \ge 4 (k \in \mathbb{N})$ \\
        $\Rightarrow (x_1 + x_2 + ... + x_k)^2 \ge 4(x_1x_2 + x_2x_3 + ... + x_kx_1)$ $(2)$ \\
        \pause
        $\Rightarrow$ Ta cần chứng minh $(2)$ đúng với $n = k + 1$. \\
        $(x_1 + x_2 + ... + x_k + x_{k + 1})^2 \ge 4(x_1x_2 + x_2x_3 + ... + x_kx_{k + 1} + x_{k + 1}x_1)$
\end{itemize}
\end{block}

\end{frame}

\begin{frame}{1.8 PHƯƠNG PHÁP QUY NẠP TOÁN HỌC \hspace{2cm}  72. Hà Anh Thư} 
\begin{block}{Các bài toán phổ thông sử dụng phương pháp quy nạp toán học}
\begin{itemize}
    \item Vì tổng hai vế của bất đẳng thức này là vòng tròn theo chỉ số, nên ta có thể giả thiết $x_{k + 1} \le x_i, i = \overline{\rm 1, k}$.
    \pause
    \item Ta có $(x_1 + x_2 + ... + x_k + x_{k + 1})^2 = (x_1 + x_2 + ... + (x_k + x_{k + 1}))^2$ \\
        $\ge 4[x_1x_2 + x_2x_3 + ... + x_{k - 1}(x_k + x_{k + 1}) + (x_k + x_{k + 1})x_1]$
    \pause
    \item Mà $[x_1x_2 + x_2x_3 + ... + x_{k - 1}(x_k + x_{k + 1}) + (x_k + x_{k + 1})x_1]$ \\
        $= (x_1x_2 + x_2x_3 + ... + x_kx_{k + 1} + x_{k + 1}x_1) + x_{k - 1}x_{k+1} + x_k(x_1 - x_{k + 1})$ \\
    \pause
    \item Vì $x_i \ge 0$ và $x_1 - x_{k + 1} \ge 0$, nên ta có \\
        $[x_1x_2 + x_2x_3 + ... + x_{k - 1}(x_k + x_{k + 1}) + (x_k + x_{k + 1})x_1]$ \\
        $\ge (x_1x_2 + x_2x_3 + ... + x_kx_{k + 1} + x_{k + 1}x_1)$.
    \pause
    \item Vậy $(x_1 + x_2 + ... + x_k + x_{k + 1})^2 \ge 4(x_1x_2 + x_2x_3 + ... + x_kx_{k + 1} + x_{k + 1}x_1)$
        $\Rightarrow$ Bất đẳng thức đúng với $n = k + 1$ $\Rightarrow$ Điều cần chứng minh.
\end{itemize}
\end{block}
\end{frame}
