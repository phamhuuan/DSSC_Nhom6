\begin{frame}{1.7 PHƯƠNG PHÁP LƯỢNG GIÁC \hspace{3cm}  71. Bùi Anh Thư}  
%\framesubtitle{} 
\begin{block}{Một số dấu hiệu}
  +) Nếu $x^2+y^2=r^2, r>0$ thì ta đặt  $\left\{
    \begin{array}{cc}
        x=rcos\alpha\\
        y=rsin\alpha
    \end{array}
    \right.$
    với \alpha \in $[0;2\pi]$ \\
+) Nếu $\frac{x^2}{a^2}+\frac{y^2}{b^2}=r^2, a,b,r>0$ thì ta đặt $\left\{
    \begin{array}{cc}
        x=racos\alpha\\
        y=rbsin\alpha
    \end{array}
    \right.$
    với \alpha \in $[0;2\pi]$ \\
    +) Nếu $\frac{x^2}{a^2}+\frac{y^2}{b^2} \leq 1, a,b,>0$ thì ta đặt $\left\{
    \begin{array}{cc}
        x=racos\alpha\\
        y=rbsin\alpha
    \end{array}
    \right.$
    với $0\leq r\leq 1$ và \alpha \in $[0;2\pi]$ .\\
    \end{block} 
\end{frame}

\begin{frame}{1.7 PHƯƠNG PHÁP LƯỢNG GIÁC \hspace{3cm}  71. Bùi Anh Thư}  
%\framesubtitle{} 
\begin{block}{Một số dấu hiệu}
    +) Nếu $|x|\leq r$ thì đặt $x=rcos\alpha $ với \alpha \in $[0;\pi]$ hoặc $x=rsin\alpha$ với $\alpha \in [\frac{-\pi}{2};\frac{\pi}{2}]$\\
    \vspace{0,4cm}
    
    +) Nếu $|x|\geq r >0$ hoặc bài toán có chứa biểu thức $\sqrt{x^2+r^2}$ thì đặt $x=\frac{r}{cos\alpha}$ với $\alpha \in [0;\frac{\pi}{2}) \cup (\frac{\pi}{2};\pi]$ hoặc $x=\frac{r}{sin\alpha}$ với $\alpha \in (0;\pi)$\\
    \vspace{0,4cm}
    
    +) Nếu biến số $x$ không có điều kiện ràng buộc thì đặt $x=tan\alpha$ với $\alpha \in (-\frac{\pi}{2};\frac{\pi}{2})$.
\end{block} 
\end{frame}

\begin{frame}{1.7 PHƯƠNG PHÁP LƯỢNG GIÁC \hspace{3cm}  71. Bùi Anh Thư} 
%\framesubtitle{} 
\begin{block}{Bài toán 1}
Cho $x^2+y^2=2$. Chứng minh rằng $2(x^3-y^3)-3(x-y)\leq 2$\\
\end{block} \\
\pause
\vspace{0,2cm}

Giải:\\
Do $x^2+y^2=2$ nên ta có thể đặt $\left\{
    \begin{array}{cc}
        x=\sqrt{2}cos\alpha\\
        y=\sqrt{2}sin\alpha
    \end{array}
    \right.$ với $\alpha\in[0;2\pi]$\\
    \vspace{0,4cm}
    
    Khi đó $2(x^3-y^3)-3(x-y)=4\sqrt{2}(\cos^3{\alpha}-\sin^3{\alpha})-3\sqrt{2}(\cos{\alpha}-\sin{\alpha})$\\
    \vspace{0,4cm}
    $=\sqrt{2}[(4\cos^3{\alpha}-3\cos{\alpha})+(3\sin{\alpha}-4\sin^3{\alpha})]$\\
    \vspace{0,4cm}
    $=\sqrt{2}(\cos{3\alpha}+\sin{3\alpha})=2\sin{3\alpha+\frac{\pi}{4}}\leq 2$\\
    \vspace{0,4cm}
    
    Ta có điều phải chứng minh.
      \end{frame}
    

 \begin{frame}{1.7 PHƯƠNG PHÁP LƯỢNG GIÁC \hspace{3cm}  71. Bùi Anh Thư}  
%\framesubtitle{} 
\begin{block}{Đổi biến số đưa về bất đẳng thức tam giác}
   Nếu bài toán cho $x,y,x$ dương thỏa mãn $xy+yz+zx=1$ hoặc $x+y+z+2xyz=1$ hoặc biến đổi đưa về ràng buộc đó thì sử dụng các kết quả sau kết hợp với các BĐT cơ bản trong tam giác.\\
   \vspace{0,4cm}
   
   \textbf{Kết quả 1:} Với $x,y,z$ là các số thực dương thỏa mãn điều kiện $xy+yz+zx=1$ khi đó:\\
   \vspace{0,4cm}
   
   1.1. Tồn tại tam giác ABC sao cho\\
   $x=\tan \frac{A}{2}; y=\tan \frac{B}{2}; z= \tan \frac{C}{2}$.\\
   \vspace{0,4cm}
   
   1.2. Tồn tại tam giác nhọn ABC sao cho\\
   $x=\cot{A}; y=\cot{B}; z=\cot{C}$.\\
\end{block} 
\end{frame}
   
   \begin{frame}{1.7 PHƯƠNG PHÁP LƯỢNG GIÁC \hspace{3cm}  71. Bùi Anh Thư} 
%\framesubtitle{} 
\begin{block}{Đổi biến số đưa về bất đẳng thức tam giác}
   \textbf{Kết quả 2.} Với $x,y,z$ là các số thực dương thỏa mãn điều kiện $x+y+z=xyz$, khi đó:\\
   \vspace{0,4cm}
   
   2.1. Tồn tại tam giác nhọn ABC sao cho\\ $x=\tan A; y=\tan B; z= \tan C$.\\
   \vspace{0,4cm}
   
   2.2. Tồn tại tam giác ABC sao cho\\ $x=\cot{\frac{A}{2}}, y=\cot{\frac{B}{2}}; z=\cot{\frac{C}{2}}$\\
\end{block} 
\end{frame}

\begin{frame}{1.7 PHƯƠNG PHÁP LƯỢNG GIÁC \hspace{3cm}  71. Bùi Anh Thư} 
%\framesubtitle{} 
\begin{block}{Đổi biến số đưa về bất đẳng thức tam giác}
   \textbf{Kết quả 3.} Với $x,y,z$ là các số thực dương thỏa mãn điều kiện $x^2+y^2+z^2+2xyz=1$, khi đó:\\
   \vspace{0,4cm}
   
   3.1. Tồn tại tam giác nhọn ABC sao cho\\ $x=\cos{A}; y=\cos{B}; z= \cos{C}$.\\
   \vspace{0,4cm}
   
   3.2. Tồn tại tam giác ABC sao cho\\ $x=\sin{\frac{A}{2}}, y=\sin{\frac{B}{2}}; z=\sin{\frac{C}{2}}$\\
\end{block} 
\end{frame}


\begin{frame}{1.7 PHƯƠNG PHÁP LƯỢNG GIÁC \hspace{3cm}  71. Bùi Anh Thư} 
%\framesubtitle{} 
\begin{block}{Bài toán 2 (Poland 1999)}
   Cho $x,y,z>0$ thỏa mãn $x+y+z=1$. Chứng minh rằng $x^2+y^2+z^2+2\sqrt{3xyz}\leq 1$\\
   \end{block}
   \pause
   \vspace{0,2cm}
   
   Giải:\\
   Ta có: $x+y+z=1$\\
   $\leftrightarrow \sqrt{\frac{yz}{x}}.\sqrt{\frac{zx}{y}}+\sqrt{\frac{zx}{y}}.\sqrt{\frac{xy}{z}}+\sqrt{\frac{xy}{z}}.\sqrt{\frac{yz}{x}}=1$\\
   \vspace{0,4cm}
   
   Sử dụng kết quả 1.1, tồn tại tam giác ABC thỏa mãn:\\
    $\left\{
    \begin{array}{ccc}
        \sqrt{\frac{yz}{x}}=\tan\frac{A}{2}\\
        \sqrt{\frac{zx}{y}}=\tan\frac{B}{2}\\
        \sqrt{\frac{xy}{z}}=\tan\frac{C}{2}
        
    \end{array}
    \right.$ 
\end{frame}

\begin{frame}{1.7 PHƯƠNG PHÁP LƯỢNG GIÁC \hspace{3cm}  71. Bùi Anh Thư} 
%\framesubtitle{} 
Khi đó, BĐT cần chứng minh tương đương với\\
$\tan^2\frac{A}{2}.\tan^2\frac{B}{2}+\tan^2\frac{B}{2}.\tan^2\frac{C}{2}+\tan^2\frac{C}{2}.\tan^2\frac{A}{2}+2\sqrt{3}\tan\frac{A}{2}\tan\frac{B}{2}\tan\frac{C}{2}\leq 1$\\
\vspace{0,4cm}

$\leftrightarrow (\tan\frac{A}{2}\tan\frac{B}{2}+\tan\frac{B}{2}\tan\frac{C}{2}+\tan\frac{C}{2}\tan\frac{A}{2})^2+2\sqrt{3}\tan\frac{A}{2}\tan\frac{B}{2}\tan\frac{C}{2}\leq 1+2\tan\frac{A}{2}\tan\frac{B}{2}\tan\frac{C}{2}(\tan\frac{A}{2}+\tan\frac{B}{2}+\tan\frac{C}{2})$\\
\vspace{0,4cm}

$\leftrightarrow \tan\frac{A}{2}+\tan\frac{B}{2}+\tan\frac{C}{2}\geq \sqrt{3}$\\

\end{frame}
