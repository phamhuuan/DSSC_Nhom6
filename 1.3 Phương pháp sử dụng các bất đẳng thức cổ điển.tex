% 1.3 Phương pháp sử dụng các bất đẳng thức cổ điển
\begin{frame}{1.3 PHƯƠNG PHÁP SỬ DỤNG CÁC \\ BẤT ĐẲNG THỨC CỔ ĐIỂN \hspace{4.5cm}  67. Đỗ Thị Thim} 
%\framesubtitle{} 
\begin{block}{Bất đẳng thức Cauchy:}
\textbf{Với n số không âm $a_1,a_2,...,a_n$ ta có $\dfrac{a_1+a_2+...a_n}{n}\geq\sqrt[n]{a_1a_2..,a_n}$\\Dấu bằng xảy ra khi $a_1=a_2=...=a_n$.}
\end{block} 
\end{frame} 

% 1.3 Phương pháp sử dụng các bất đẳng thức cổ điển
\begin{frame}{1.3 PHƯƠNG PHÁP SỬ DỤNG CÁC \\ BẤT ĐẲNG THỨC CỔ ĐIỂN \hspace{4.5cm}  67. Đỗ Thị Thim} 
%\framesubtitle{} 
\begin{block}{Bài tập:}
\textbf{Cho $a,b,c\geq$ 0. Chứng minh rằng:\\$(a+b+c)^3\leq$9($a^3+b^3+c^3$)}
\end{block} 
\end{frame} 

% 1.3 Phương pháp sử dụng các bất đẳng thức cổ điển
\begin{frame}{1.3 PHƯƠNG PHÁP SỬ DỤNG CÁC \\ BẤT ĐẲNG THỨC CỔ ĐIỂN \hspace{4.5cm}  67. Đỗ Thị Thim} 
%\framesubtitle{} 
\begin{block}{Lời giải:}
\textbf{$(a+b+c)^3=a^3+b^3+c^3+3(a+b)(b+c)(c+a)$\\ Áp dụng bất đẳng thức Cauchy cho a+b,b+c,c+a ta có:\\$(a+b+c)^3\leq a^3+b^3+c^3+3(\dfrac{a+b+b+c+c+a}{3})^3$\\$\Leftrightarrow 9(a+b+c)^3\leq 9(a^3+b^3+c^3)+8(a+b+c)^3$\\$\Leftrightarrow (a+b+c)^3\leq 9(a^3+b^3+c^3)$.}
\end{block}
\end{frame} 

% 1.3 Phương pháp sử dụng các bất đẳng thức cổ điển
\begin{frame}{1.3 PHƯƠNG PHÁP SỬ DỤNG CÁC \\ BẤT ĐẲNG THỨC CỔ ĐIỂN \hspace{4.5cm}  67. Đỗ Thị Thim} 
%\framesubtitle{} 
\begin{block}{Bài tập tương tự:}
\textbf{$Cho a,b,c>0, a+b+c=\dfrac{3}{4}$. Chứng minh rằng:\\a.A=$\sqrt[3]{a+3b}+\sqrt[3]{b+3c}+\sqrt[3]{c+3a}\leq3$\\$b.B=\sqrt[3]{a+7b}+\sqrt[3]{b+7c}+\sqrt[3]{c+7a}\leq 3\sqrt[3]{2}$.}
\end{block} 
\end{frame} 

% 1.3 Phương pháp sử dụng các bất đẳng thức cổ điển
\begin{frame}{1.3 PHƯƠNG PHÁP SỬ DỤNG CÁC \\ BẤT ĐẲNG THỨC CỔ ĐIỂN \hspace{4.5cm}  67. Đỗ Thị Thim} 
%\framesubtitle{} 
\begin{block}{Bất đẳng thức Bunhiakovski(Cauchy-Schwarz):}
      \hspace{0.5cm}
       Giả sử $a_1,a_2,...,a_n$ và $b_1,b_2,...,b_n$ là các số thực, ta có:\\$(a_1^{2}+a_2^{2}+...+a_n^{2})(b_1^{2}+b_2^{2}+...+b_n^{2})\geq(a_1b_1+a_2b_2+...+a_nb_n)^2$
       \end{block}
       \begin{block}{Bất đẳng thức Svacxơ:}
            \hspace{0.5cm}
        Cho 2 dãy số thực $a_1,a_2,...,a_b$ và $ b_1,b_2,...,b_n(b_i>0,i=1,2,..,n)$ ta có:\\$\dfrac{a_1^{2}}{b_1}+\dfrac {a_2^{2}}{b_2}+...+\dfrac{a_n^{2}}{b_n}\geq\dfrac{(a_1+a_2+...+a_n)^2}{b_1+b_2+...+b_n}$
\end{block} 
\end{frame} 

% 1.3 Phương pháp sử dụng các bất đẳng thức cổ điển
\begin{frame}{1.3 PHƯƠNG PHÁP SỬ DỤNG CÁC \\ BẤT ĐẲNG THỨC CỔ ĐIỂN \hspace{4.5cm}  67. Đỗ Thị Thim} 
%\framesubtitle{} 
\begin{block}{Bất đẳng thức Chebyshev:}
\textbf{Với 2 dãy $a_1\leq a_2\leq ...\leq a_n$ và $b_1\leq b_2\leq...\leq b_n$, ta có bất đẳng thức:\\$n(a_1b_1+a_2b_2+...+a_nb_n)\geq (a_1+a_2+...+a_n)(b_1+b_2+..+b_n)$}
\end{block} 
\end{frame} 

% 1.3 Phương pháp sử dụng các bất đẳng thức cổ điển
\begin{frame}{1.3 PHƯƠNG PHÁP SỬ DỤNG CÁC \\ BẤT ĐẲNG THỨC CỔ ĐIỂN \hspace{4.5cm}  67. Đỗ Thị Thim} 
%\framesubtitle{} 
\begin{block}{Bài tập:}
\textbf{Cho $a,b,c,d>0$, $a^2+b^2+c^2+d^2=4$. Chứng minh:\\$\dfrac{a^2}{b+c+d}+\dfrac{b^2}{c+d+a}+\dfrac{c^2}{d+a+b}+\dfrac{d^2}{a+b+c}\geq \dfrac{4}{3}$}
\end{block} 
\end{frame} 

% 1.3 Phương pháp sử dụng các bất đẳng thức cổ điển
\begin{frame}{1.3 PHƯƠNG PHÁP SỬ DỤNG CÁC \\ BẤT ĐẲNG THỨC CỔ ĐIỂN \hspace{4.5cm}  67. Đỗ Thị Thim} 
%\framesubtitle{} 
\begin{block}{Lời giải:}
\textbf Không mất tính tổng quát ta giả sử $a\geq b\geq c\geq d$\\Khi đó $a^2\geq b^2\geq c^2\geq d^2$\\ $b+c+d<c+d+a<d+a+b<a+b+c\Rightarrow\dfrac{1}{b+c+d}$>$\dfrac{1}{c+d+a}$>$\dfrac{1}{d+a+b}$>$\dfrac{1}{a+b+c}$
\end{block} 
\end{frame} 

% 1.3 Phương pháp sử dụng các bất đẳng thức cổ điển
\begin{frame}{1.3 PHƯƠNG PHÁP SỬ DỤNG CÁC \\ BẤT ĐẲNG THỨC CỔ ĐIỂN \hspace{4.5cm}  67. Đỗ Thị Thim} 
       Sử dụng bất đẳng thức Chebyshev ta có:
       VT$\geq\dfrac{1}{4}(a^2+b^2+c^2+d^2)(\dfrac{1}{b+c+d}+\dfrac{1}{c+d+a}+\dfrac{1}{d+a+b}+\dfrac{1}{a+b+c})$\\ Sử dụng bất đẳng thức Svacsơ ta có:\\ VT$\geq\dfrac{1}{4}(a^2+b^2+c^2+d^2)\dfrac{4^2}{3(a+b+c+d)}=\dfrac{4(a^2+b^2+c^2+d^2)}{3(a+b+c+d)}$
\end{frame} 

% 1.3 Phương pháp sử dụng các bất đẳng thức cổ điển
\begin{frame}{1.3 PHƯƠNG PHÁP SỬ DỤNG CÁC \\ BẤT ĐẲNG THỨC CỔ ĐIỂN \hspace{4.5cm}  67. Đỗ Thị Thim} 
       Ta có$(a-1)^2+(b-1)^2+(c-1)^2+(d-1)^2\geq$ 0\\$\Leftrightarrow a^2+b^2+c^2+d^2+4\geq 2(a+b+c+d)$\\$\Leftrightarrow a+b+c+d\leq 4\Leftrightarrow\dfrac{a^2+b^2+c^2+d^2}{a+b+c+d}\geq1$\\$\Rightarrow VT\geq\dfrac{4}{3}$.Vậy ta có đpcm.
\end{frame}

