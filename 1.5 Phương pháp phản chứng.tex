% 1.5 Phương pháp phản chửng
\begin{frame}{1.5 PHƯƠNG PHÁP PHẢN CHỨNG \hspace{3cm}  69. Cao Thị Thùy} 
%\framesubtitle{} 
\begin{block}{Phương pháp phản chứng}
\textbf{Định nghĩa:}  Sử dụng \textit{phương pháp phản chứng} là đi tìm sự mâu thuẫn từ giả thuyết đến kết luận, tức là nếu ta muốn chứng minh kết luận của bài toán là đúng thì phải chứng minh cái ngược lại với nó là sai.
    
\end{block} 
\end{frame} 
\begin{frame}{1.5 PHƯƠNG PHÁP PHẢN CHỨNG \hspace{3cm}  69. Cao Thị Thùy} 
%\framesubtitle{} 
\begin{block}{Ví dụ}
1. Chứng minh rằng với mọi số tự nhiên $n$ mà $n^3$ chia hết cho 3 thì $n$ chia hết cho 3.\\
2. Cho các số $a,b,c,d$ phân biệt. Chứng minh rằng trong 4 đẳng thức:\\ $a^2+3b^2-3cd\leqslant 0$  (1) \\$b^2+3c^2-3da \leqslant 0$  (2) \\ $c^2+3d^3-3ab\leqslant 0$  (3) \\$d^2+3a^2-3bc\leqslant 0$   (4) \\
3. Cho $a,b,c$ là các số dương thỏa mãn $abc=1$. Chứng minh rằng nếu $a+b+c> \frac{1}{a}+\frac{1}{b}+\frac{1}{c}$ thì có một và chỉ một trong ba số $a,b,c$ lớn hơn 1.
\end{block} 
\end{frame} 
\begin{frame}{1.5 PHƯƠNG PHÁP PHẢN CHỨNG \hspace{3cm}  69. Cao Thị Thùy} 
%\framesubtitle{} 
\begin{block}{Bài toán 1}
Chứng minh rằng với mọi số tự nhiên $n$ mà $n^3$ chia hết cho 3 thì $n$ chia hết cho 3.
\end{block}
\begin{center}
    \textit{Lời giải:}
\end{center}
\begin{block}{}
    Giả sử $n$ không chia hết cho 3, khi đó $n=3k+1$ hoặc $n=3k+2, k \in Z$ \\
    Với $n=3k+1$ ta có $n^3=(3k+1)^3=27k^3+27k^2+9k+1$ không chia hết cho 3 (mâu thuẫn).\\
    Với  $n=3k+2$ ta có $n^3=(3k+2)^3=27k^3+54k^2+36k+4$ không chia hết cho 3 (mâu thuẫn).\\
    Vậy $n$ chia hết cho 3.
\end{block} 
\end{frame}
\begin{frame}{1.5 PHƯƠNG PHÁP PHẢN CHỨNG \hspace{3cm}  69. Cao Thị Thùy} 
%\framesubtitle{} 
\begin{block}{Bài   2}
Cho các số $a,b,c,d$ phân biệt. Chứng minh rằng trong 4 đẳng thức:\\ $a^2+3b^2-3cd\leqslant 0$  (1) \\$b^2+3c^2-3da \leqslant 0$  (2) \\ $c^2+3d^3-3ab\leqslant 0$  (3) \\$d^2+3a^2-3bc\leqslant 0$   (4) \\
\end{block} 
\end{frame} 
\begin{block}{Bài toán 2}
Cho các số $a,b,c,d$ phân biệt. Chứng minh rằng trong 4 đẳng thức:\\ $a^2+3b^2-3cd\leqslant 0$  (1) \\$b^2+3c^2-3da \leqslant 0$  (2) \\ $c^2+3d^3-3ab\leqslant 0$  (3) \\$d^2+3a^2-3bc\leqslant 0$   (4) \\
\textit{Lời giải:}
Ta cần chứng minh trong 2 bất đẳng thức (1) và (3) phải có một bất đẳng thức sai.\\
Giả sử cả (1) và (3) đều đúng, khi đó \\
$(a^2+3b^2-3cd)+(c^2+3d^2-3ab) \leqslant 0 $\\ $\Leftrightarrow (a^2+3b^2-3ab)+(c^2+3d^2-3cd) \leqslant 0 $\\ 
$\Leftrightarrow (a-\frac{3}{2}b)^2+\frac{3}{4}b^2+(c-\frac{3}{2}d)^2+\frac{3}{4}d^2 \leqslant 0$\\  $\Leftrightarrow (a-\frac{3}{2}b)^2=\frac{3}{4}b^2=(c-\frac{3}{2}d)^2=\frac{3}{4}d^2$\\ 
$\Leftrightarrow a=b=c=d=0$ (mâu thuẫn).\\
Nên trong hai bất đẳng thức (1) và (3) phải có ít nhất một bất đẳng thức sai.
\end{block} 
\end{frame} 
\begin{frame}{1.5 PHƯƠNG PHÁP PHẢN CHỨNG \hspace{3cm}  69. Cao Thị Thùy} 
%\framesubtitle{} 
\begin{block}{Bài toán 2}
\textit{Lời giải:}
Giả sử cả (2) và (4) đều đúng, khi đó \\
$(b^2+3c^2-3da)+(d^2+3a^2-3bc) \leqslant 0 $\\ $\Leftrightarrow (b^2+3c^2-3bc)+(d^2+3d^2-3cd) \leqslant 0 $\\ 
$\Leftrightarrow (b-\frac{3}{2}c)^2+\frac{3}{4}c^2+(d-\frac{3}{2}a)^2+\frac{3}{4}a^2 \leqslant 0$\\  $\Leftrightarrow (b-\frac{3}{2}c)^2=\frac{3}{4}c^2=(d-\frac{3}{2}a)^2=\frac{3}{4}a^2$\\ 
$\Leftrightarrow a=b=c=d=0$ (mâu thuẫn).\\
Nên trong hai bất đẳng thức (2) và (4) phải có ít nhất một bất đẳng thức sai.\\
Vậy trong bốn bất đẳng thức đã cho có ít nhất hai bất đẳng thức sai.
\end{block} 
\end{frame} 
\begin{frame}{1.5 PHƯƠNG PHÁP PHẢN CHỨNG \hspace{3cm}  69. Cao Thị Thùy} 
%\framesubtitle{} 
\begin{block}{Mở rộng}
1. Có hay không số tự nhiên $n$ để $2010+n^2$ là số chính phương. \\
2. Cho các số $a,b,c,d>0$. Chứng minh rằng trong bốn bất đẳng thức sau, có ít nhất hai bất đẳng thức sai.\\
\begin{table}[]
    \centering
    \begin{tabular}{ccc}
       $2a+b-2\sqrt{cd} \leqslant 0$,  && $2b+c-2\sqrt{da}\leqslant 0$ \\
        $2c+d-2\sqrt{ab}\leqslant 0$, && $2d+a-2\sqrt{bc}\leqslant 0$
    \end{tabular}
    \label{tab:my_label}
\end{table}
3. Chứng minh $\sqrt{2}$ là số vô tỉ.
\end{block} 
\end{frame} 