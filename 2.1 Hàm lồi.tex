% 2.1 Hàm lồi
\begin{frame}{2.1 HÀM LỒI. \hspace{6cm}  75. Nguyễn Thị Thu Trang.} 
%\framesubtitle{} 
\begin{block}{Khái niệm hàm lồi: }
Hàm $y=f(x)$ được gọi là \textit{hàm lồi (lồi xuống dưới)} trong khoảng $(a; b)$ nếu với mọi $x_1, x_2 \in (a; b)$ và mọi số thực $\alpha_1 , \alpha_2 \ge 0, \alpha_1 +\alpha_2 = 1 $, ta có:
\begin{center}
    $\alpha_1 f(x_1) +\alpha_2 f(x_2) \ge f(\alpha_1 x_1 + \alpha_2 x_2).$
\end{center}
Hàm $y=f(x)$ được gọi là \textit{hàm lõm} trong khoảng $(a; b)$ nếu $y=-f(x)$ là hàm lồi.
\end{block} 
\end{frame} 
\begin{frame}{2.1 HÀM LỒI. \hspace{6cm}  75. Nguyễn Thị Thu Trang.} 
%\framesubtitle{} 
\begin{block}{Ví dụ 1: }
Hàm $f: \mathbb{R} \longrightarrow \mathbb{R}$\\
\hspace{1cm} $x\longmapsto f(x)= x^2$ là hàm lồi.
\end{block} 
\begin{center}
    \includegraphics[scale=0.25]{Screenshot (373).png}
\end{center}
\end{frame}
\begin{frame}{2.1 HÀM LỒI. \hspace{6cm}  75. Nguyễn Thị Thu Trang.} 
%\framesubtitle{} 
\begin{block}{Ví dụ 1: }
Hàm $f: \mathbb{R} \longrightarrow \mathbb{R}$\\
\hspace{1cm} $x\longmapsto f(x)= x^2$ là hàm lồi.
\end{block} 
Xét $x,y \in \mathbb{R}; \alpha \in [0; 1]$ ta có: \\$f(\alpha x + (1- \alpha)y) - \alpha f(x) - (1-\alpha) f(y)$\\
\vspace{2cm}
$f(\alpha x + (1- \alpha)y)\leq \alpha f(x) + (1-\alpha) f(y)$ với mọi $x, y \in \mathbb{R}, \alpha \in [0; 1]$\\
$\Rightarrow$  Hàm $f$ lồi.
\end{frame}
\begin{frame}{2.1 HÀM LỒI .\hspace{6cm}  75. Nguyễn Thị Thu Trang.} 
%\framesubtitle{} 
\begin{block}{Ví dụ 1: }
Hàm $f: \mathbb{R} \longrightarrow \mathbb{R}$\\
\hspace{1cm} $x\longmapsto f(x)= x^2$ là hàm lồi.
\end{block} 
Xét $x,y \in \mathbb{R}; \alpha \in [0; 1]$ ta có: \\$f(\alpha x + (1- \alpha)y) - \alpha f(x) - (1-\alpha) f(y)$\\
\pause
$= [\alpha x + (1- \alpha) y ]^2 - \alpha x^2 - (1-\alpha) y^2$ \\
\pause
$= (\alpha^2 -\alpha) x^2 + 2\alpha (1-\alpha) xy + (\alpha^2-\alpha) y^2$\\
\pause
$= \alpha(\alpha-1)(x^2- 2xy+y^2)$\\
\pause
$=\alpha(\alpha-1)(x-y)^2 \leq 0$\\
$\Rightarrow f(\alpha x + (1- \alpha)y)\leq \alpha f(x) + (1-\alpha) f(y)$ với mọi $x, y \in \mathbb{R}, \alpha \in [0; 1]$\\
$\Rightarrow$  Hàm $f$ lồi.
\end{frame}
\begin{frame}{2.1 HÀM LỒI. \hspace{6cm}  75. Nguyễn Thị Thu Trang.}
\begin{block}{Ví dụ 2: }
Hàm $h: \mathbb{R} \longrightarrow \mathbb{R}$\\
\hspace{1.2cm} $x\longmapsto h(x)= x^3$ \textbf{không} là hàm lồi.
\end{block}
\begin{center}
    \includegraphics[scale=0.5]{1561336492492_5.png}
\end{center}
\end{frame}
\begin{frame}{2.1 HÀM LỒI. \hspace{6cm}  75. Nguyễn Thị Thu Trang.}
\begin{block}{Ví dụ 2: }
Hàm $h: \mathbb{R} \longrightarrow \mathbb{R}$\\
\hspace{1.2cm} $x\longmapsto h(x)= x^3$ \textbf{không} là hàm lồi.
\end{block}
\textbf{Mục tiêu:} Chỉ ra \textbf{tồn tại} $x,y \in \mathbb{R}, \alpha \in [0,1]$ thỏa mãn: $h(\alpha x+(1 -\alpha)y) > \alpha h(x) + (1-\alpha)h(y)$\\
\end{frame}
\begin{frame}{2.1 HÀM LỒI. \hspace{6cm}  75. Nguyễn Thị Thu Trang.}
\begin{block}{Ví dụ 2: }
Hàm $h: \mathbb{R} \longrightarrow \mathbb{R}$\\
\hspace{1.2cm} $x\longmapsto h(x)= x^3$ \textbf{không} là hàm lồi.
\end{block}
Với $x = -3, y = 1, \alpha = \dfrac{1}{2}$ ta có:\\
$h(\alpha x+(1 -\alpha)y) = h(-1)=-1 $\\
$\alpha h(x) + (1-\alpha)h(y)= \dfrac{1}{2}(-3)^3 +\dfrac{1}{2}. 1^3 = -13 $\\
$\Rightarrow  \exists x, y \in \mathbb{R}, \alpha \in [0; 1]:  h(\alpha x+(1 -\alpha)y) > \alpha h(x) + (1-\alpha)h(y) $\\
$\Rightarrow h$ không là hàm lồi.
\end{frame}
\begin{frame}{2.1 HÀM LỒI. \hspace{6cm}  75. Nguyễn Thị Thu Trang.}
\begin{block}{Ví dụ 3: }
Hàm $g: \mathbb {R^+} \longrightarrow \mathbb {R}$\\
\hspace{1.25cm} $x\longmapsto g(x)= x^3$ là hàm lồi.
\end{block}
\pause
Xét $x,y \in \mathbb{R^+}; \alpha \in [0; 1]$ ta có: \\$g(\alpha x + (1- \alpha)y) - \alpha g(x) - (1-\alpha) g(y) = [\alpha x + (1- \alpha) y ]^3 - \alpha x^3 - (1-\alpha) y^3$ \\
\pause
$=(\alpha^3 -\alpha)x^3 +[(1-\alpha)^3 - (1-\alpha)]y^3 + 3\alpha^2 (1- \alpha) x^2 y + 3\alpha(1-\alpha)^2 xy^2$\\
$=\alpha(\alpha -1 )[(\alpha+1) x^3 + (2-\alpha)y^3 -3\alpha x^2y -3(1-\alpha)xy^2]  $\\
$=\alpha(\alpha-1)[\alpha(x^3-3x^2y+3xy^2-y^3)+x^3 -3xy^2 +2y^3]$\\
$=\alpha(\alpha-1)[\alpha(x-y)^3+(x-y)^2 (x+2y)]$\\
$=\alpha(\alpha-1)(x-y)^2[\alpha(x-y) +x+2y]$\\
$=\alpha(\alpha-1)(x-y)^2[(\alpha+1)x+(2-\alpha)y] \leq 0$\\
$\Rightarrow g(\alpha x + (1- \alpha)y)\leq \alpha g(x) + (1-\alpha) g(y)$ với mọi $x, y \in  \mathbb {R^+}, \alpha \in [0; 1]$\\
$\Rightarrow$  Hàm $g$ lồi.
\end{frame}
\begin{frame}{2.1 HÀM LỒI. \hspace{6cm}  75. Nguyễn Thị Thu Trang.}
\begin{block}{Định lý 1:}
Hàm $y=f(x)$ xác định, liên tục và có đạo hàm $f'(x)$ hữu hạn trên khoảng $(a; b)$, là hàm lồi trong khoảng $(a; b)$ khi và chỉ khi $f'(x)$ đồng biến trong $(a; b).$
\end{block}
\pause
\begin{block}{Định lý 2:}
Hàm $y=f(x)$ xác định, liên tục và có đạo hàm $f'(x)$ hữu hạn trên khoảng $(a; b)$, là hàm lồi trong khoảng $(a; b)$ khi và chỉ khi $f"(x) \ge 0$ thỏa mãn cho mọi $ x \in (a; b).$
\end{block}
\end{frame}
\begin{frame}{2.1 HÀM LỒI. \hspace{6cm}  75. Nguyễn Thị Thu Trang.}
\begin{minipage}{0.5\linewidth}
\begin{block}{Định lý 3: }
    Cho hàm số $y=f(x)$ xác định liên tục và có đạo hàm $f'(x)$ hữu hạn trong khoảng $(a; b)$. Khi đó nếu $f(x)$ là hàm lồi trong khoảng $(a; b)$ thì $f(x)-f(y) \ge (x-y)f'(y)$ thỏa mãn cho mọi $x; y$ thuộc khoảng $(a; b)$.
\end{block}
\end{minipage}\qquad
 \begin{minipage}{0.4\linewidth}
			\includegraphics[scale=0.5]{Screenshot (374).png}
\end{minipage}
\end{frame}
\begin{frame}{2.1 HÀM LỒI. \hspace{6cm}  75. Nguyễn Thị Thu Trang.}
\begin{block}{Bài toán 1:}
Với $3$ số thực dương $a; b; c$ thỏa mãn: $a+b+c=3$. Chứng minh rằng:\\
$$\frac{a}{3-a}+\frac{b}{3-b}+\frac{c}{3-c}\ge \frac{3}{2}$$
\end{block}
\pause
\textbf{Phân tích:} \\
- Trước tiên ta có nhận xét được đẳng thức xảy ra khi và chỉ khi $a=b=c=1$.\\
\pause
- Ta thấy $f(x)= \dfrac{x}{3-x}$ là hàm lồi với mọi $0 < x < 3$
 \begin{minipage}{0.4\linewidth}
			\includegraphics[scale=0.25]{Screenshot (375).png}
\end{minipage}\\
\pause
$\Rightarrow $ Ta có: $f(a)-f(1)\ge (a-1)f'(1) \Rightarrow \dfrac{a}{3-a}\ge \dfrac{3}{4}(a-1)+\dfrac{1}{2}$
\end{frame}
\begin{frame}{2.1 HÀM LỒI. \hspace{6cm}  75. Nguyễn Thị Thu Trang.}
\begin{block}{Bài toán 1:}
Với $3$ số thực dương $a; b; c$ thỏa mãn: $a+b+c=3$. Chứng minh rằng:\\
$$\frac{a}{3-a}+\frac{b}{3-b}+\frac{c}{3-c}\ge \frac{3}{2}$$
\end{block}
\textbf{Lời giải:}\\
Xét hàm số: $f(x)= \dfrac{x}{3-x} \Rightarrow f'(x)=\dfrac{3}{(3-x)^2} \Rightarrow f"(x)= \dfrac{2}{(3-x)^3} > 0$ với mọi $0< x< 3$\\
$\Rightarrow f(x)=\dfrac{x}{3-x}$ là hàm lồi trong khoảng $(0; 3) $ nên ta có:\\
 $f(a)-f(1)\ge (a-1)f'(1) \Rightarrow \dfrac{a}{3-a}\ge \dfrac{3}{4}(a-1)+\dfrac{1}{2}$\\
 Tương tự ta có: $ \dfrac{b}{3-b}\ge \dfrac{3}{4}(b-1)+\dfrac{1}{2}$ và $\dfrac{c}{3-c}\ge \dfrac{3}{4}(c-1)+\dfrac{1}{2}$\\
 $\Rightarrow \dfrac{a}{3-a}+\dfrac{b}{3-b}+\dfrac{c}{3-c} \ge \dfrac{3}{4}(a+b+c-3)+\dfrac{3}{2 }=\dfrac{3}{2} $
\end{frame}
\begin{frame}{2.1 HÀM LỒI. \hspace{6cm}  75. Nguyễn Thị Thu Trang.}
\begin{block}{Bài toán 2 (Baltic way 2011)}
Với bốn số thực $a; b; c; d$ không âm có tổng bằng $4$. Chứng minh rằng:\\
$$\frac{a}{a^3+8}+\frac{b}{b^3+8}+\frac{c}{c^3+8}+\frac{d}{d^3+8} \leq \frac{4}{9}$$
\end{block}
\begin{block}{Bài toán 3 (All- Rusian Olympiad 2002)}
Với ba số thực dương $a; b; c$ có tổng bằng$3$.Chứng minh rằng:\\
$$\sqrt{a}+\sqrt{b}+\sqrt{c} \ge ab +bc +ca$$
\end{block}
\textbf{*Gợi ý bài 3:} Đưa biểu thức cần chứng minh về:\\
$$a^2 + 2\sqrt{a}+b^2 +2\sqrt{b}+ c^2 +2\sqrt{c} \ge 2(ab+bc+ca) + a^2 + b^2 +c^2= 9$$
\end{frame}
